Questa sezione ha l'obiettivo di elencare e descrivere l'utilizzo degli strumenti utilizzati nell'ambito della validazione delle pagine.\\ 
La validazione è uno dei requisiti più sottovalutati, ma in realtà è uno dei più importanti, in quanto garantisce che il codice scritto sia corretto e di qualità, perchè conforme agli standard W3C.\\
I vantaggi della validazione di un sito web sono molteplici:
\begin{itemize}
	\item un sito è più accessibile se viene validato;
	\item se il codice è allineato con lo standard, permette di limitare le differenze di visualizzazione del sito da un browser all'altro e garantirne la compatibilità;
	\item la presenza di errori nella pagina rallenta la lettura da parte dei browser, pena un peggioramento della navigazione e dell'esperienza utente;
	\item la validazione di un sito influenza la sua indicizzazione ed il suo posizionamento sui vari motori di ricerca, pena meno traffico nel proprio sito web.  
\end{itemize}
Gli strumenti di validazione utilizzati nel corso del progetto sono i seguenti:
\begin{itemize}
	\item \textbf{W3C HTML Validator}\\
	Indirizzo sito web: \emph{https://validator.w3.org/}\\
	Servizio di validazione gratuito di W3C che consente di validare il codice (X)HTML.\\ 
	Inizialmente, la validazione era più instantanea in quanto era presente solo codice che generava pagine statiche ((X)HTML).\\Dal momento in cui è stato scritto codice che genera pagine dinamiche (PHP), allora si è provveduto a incollare l'intero codice sorgente in una sezione dedicata nel sito.\\
	Se il codice non è valido, segnala il numero di errori, il tipo di errori e a quale riga e colonna della pagina sono stati trovati.
	\item \textbf{W3C CSS Validator}\\
	Indirizzo sito web: \emph{http://www.css-validator.org/}\\
	Servizio di valutazione gratuito di W3C che consente di validare il codice CSS.\\
	In questo caso basta scegliere il file CSS dalla directory e validarlo nell'apposita sezione.\\
	Con questo servizio è possibile validare il file (X)HTML con CSS integrato, ma non è il nostro caso in quanto, come richiesto dalle regole di progetto, \emph{il sito web deve rispettare la completa separazione tra contenuto, presentazione e comportamento}. 
\end{itemize}

 