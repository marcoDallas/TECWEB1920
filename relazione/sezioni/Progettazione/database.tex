Un'altra parte fondamentale del progetto è il database, in quanto il sito si appoggia su di esso per richiedere parti del contenuto informativo.\\ 
In fase di progettazione, è stato pensato di creare un database con la funzione di contenere tutti i dati relativi ai prodotti in vendita, ovvero \emph{paste} e \emph{torte}, 
e al contenuto del contenitore \emph{news}.\\
Come si può vedere dalla \emph{Figura \ref{Fig:schemadb}}, le tabelle (che rappresentano sià entità che associazioni e i loro relativi attributi chiave/non chiave) sono quattro:
\begin{itemize}
		\item \textbf{Prodotto:} contiene tutti i dati relativi ai prodotti che verranno visualizzati nelle pagine \emph{Paste} e \emph{Torte}
		\item \textbf{Recensione:} contiene tutti i dati riguardo al feedback rilasciato da un utente per un prodotto specifico
		\item \textbf{Utente:} contiene i dati di tutti gli utenti che sono registrati al sito della pasticceria
		\item \textbf{News:} contiene tutti i dati addetti a riempire la sezione \emph{News} sotto il menu 
\end{itemize}
\begin{figure}[!h]
	\centering
	\includegraphics[width=0.7\linewidth]{sezioni/Progettazione/Immagini/schema_concettuale.jpg}\\
    \caption{Schema concettuale database}
	\label{Fig:schemadb}
\end{figure}
Ai fini di questo progetto, per vari motivi che sono stati discussi all'interno del gruppo, tre delle quattro tabelle elencate sono utilizzate nel momento in cui PHP interviene per 
prelevare i dati di interesse dal database.\\
Come spiegato nella sezione \emph{Analisi}, esistono due tipi di utenti in questo progetto: \emph{utente generico} e \emph{amministratore}.\\ L'utente generico non viene salvato nella 
tabella \emph{Utente}, in quanto non ha bisogno di registrarsi al sito per la semplice navigazione, mentre l'amministratore deve avere le credenziali appposite per entrare 
nell'\emph{Area amministratore}.\\
Inoltre, è obbligatorio l'utilizzo della tabella \emph{News} per visualizzare le novità della pasticceria, e della tabella \emph{Prodotto} per poter visualizzare paste e torte, in quanto 
i dati sono richiesti dinamicamente (altrimenti le pagine sarebbero vuote).\\
Ergo, la tabella \emph{Recensione} può servire solo nel caso in cui si vuole aggiungere un servizio interno al sito che prevede la registrazione di utenti che possono scrivere recensioni 
per i prodotti della pasticceria. Quindi, ai fini degli obiettivi del nostro progetto, la tabella è stata creata con l'intento di averla pronta per implementazioni future.\\
Stesso discorso si applica alla tabella \emph{Utente}: in questo progetto, questa tabella sarà riempita solamente con i dati dell'amministratore, in quanto unico utente ad aver accesso 
all'\emph{Area amministratore} implementata nel sito.\\
Nel momento in cui si vorrà implementare il servizio di recensioni utente, allora la tabella sarà composta anche dai parametri degli utenti che si registrano al servizio mediante pagina 
di registrazione (non prevista in questo progetto).
