Il foglio di stile implementato garantisce un design fluido e scalabile, grazie all'utilizzo di unità di misura
sempre relative o in percentuale. \\Questo migliora l'accessibilità e garantisce una corretta visualizzazione delle pagine
su tutti i formati di schermi. \\Il sito dispone di 4 modalità di visualizzazione differente: desktop, tablet, mobile e di stampa.

\subsection{Desktop}
La versione desktop è stata pensata in modo da avere una navigazione estremamente semplice.\\ 
Nella colonna di sinistra sono presenti anche le \emph{news}, ed una sezione \emph{vieni a trovarci} con gli orari di apertura della pasticceria.\\ 
Da notare la larghezza massima impostata a 1200px, che permette agli utenti di utilizzare il sito con una finestra ridotta su schermi di certe dimensioni.

\begin{figure}[!h]
	\centering
	\includegraphics[width=1\linewidth]{sezioni/Progettazione/Immagini/desktop_example.png}
	\caption{Esempio di una pagina in versione desktop}
	\label{Fig:verDesktop}
\end{figure}
\newpage
\subsection{Tablet}
La versione tablet implementa l'interfaccia in modo diverso. Dato lo spazio limitato dello schermo, il gruppo ha deciso di implementare
un menù ad hamburger. Anche le \emph{news} non sono più laterali, ma si trovano in cima al contenuto della pagina. In questo modo viene data
maggiore importanza alle \emph{news} e agli orari della pasticceria, cosa necessaria nei dispositivi portatili.\\
Anche il form di login da parte dell'amministratore nelle versioni mobili è stato nascosto e reso disponibile tramite un bottone.
\begin{figure}[!h]		
    \centering		  
	\includegraphics[width=0.8\linewidth]{sezioni/Progettazione/Immagini/tablet_example.png}
	\caption{Esempio di una pagina in versione tablet}
	\label{Fig:verTablet}
\end{figure}	    
\newpage

\subsection{Mobile}
La versione mobile differisce di molto poco rispetto alla versione tablet. Solo \emph{news} e sezione \emph{vieni a trovarci} con orari sono state poste incolonnate invece che affiancate.
\begin{figure}[!h]
    \centering		  
	\includegraphics[width=0.8\linewidth]{sezioni/Progettazione/Immagini/mobile_example.png}
	\caption{Esempio di una pagina in versione mobile}
	\label{Fig:verMobile}
\end{figure}
\newpage

\subsection{Print}
La stampa elimina tutte le funzionalità interattive. In questo modo una pagina stampata conterrà solamente il contenuto interessato.\\
Vengono quindi eliminati la maggior parte degli stili, dei colori e delle immagini in quanto l'obbiettivo di una stampa è ottenere un foglio 
contenente solo informazioni realmente utili.\\
La scelta è stata quella di mantenere le foto ai prodotti nella stampa. Questo poichè la presentazione dei prodotti di una pasticceria è di rilevante importanza.\\
\begin{figure}[!h]
    \centering		  
	\includegraphics[width=0.8\linewidth]{sezioni/Progettazione/Immagini/print_example.png}
	\caption{Esempio di una stampa}
	\label{Fig:verPrint}
\end{figure}


\newpage