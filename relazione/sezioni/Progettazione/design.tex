La strategia di progettazione del design che è stata implementata è \emph{Responsive Web Design}.\\
Questa tecnica è incentrata sull'accessibilità e permette la realizzazione del sito in modo che si possa adattare graficamente in modo automatico al tipo di device che si sta utilizzando.\\ 
Il layout che è stato scelto è un \emph{layout a cinque pannelli}, che si adatta facilmente al concetto di \emph{Responsive Web Design}.\\
I pannelli presenti nel layout sono, in ordine di presentazione:
\begin{enumerate}
	\item \textbf{header:} contiene il logo della pasticceria e un titolo che descrive la pagina corrente;
	\item \textbf{breadcrumb:} contiene il percorso della pagina corrente, serve ad orientare l'utente all'interno del sito;
	\item \textbf{menu:} contiene le varie sezioni del sito: quella corrente appare in grassetto, mentre le sezioni già visitate sono contrassegnate in viola. Inoltre sotto le sezioni appaiono due box con informazioni relative alle news e agli orari della pasticceria;
	\item \textbf{content:} contiene il contenuto della pagina corrente;
	\item \textbf{footer:} contiene le informazioni riguardo i contatti, gli orari della pasticceria, il form per accedere all'\emph{Area Amministratore}, i progettisti del sito, il copyright e gli standard rispettati.
\end{enumerate}

\newpage
Il layout si presenta in modi diversi, in base alle dimensioni dello schermo sulla quale l'utente sta navigando:\\
\begin{itemize}
	\item
	\textbf{Versione desktop}\\ 
	\begin{figure}[!h]
		\centering
		\includegraphics[width=0.7\linewidth]{sezioni/Progettazione/Immagini/desktop_layout.png}
	    \caption{Layout di una pagina in versione desktop}
		\label{Fig:verDesktop}
	\end{figure}
	\item	  
	\textbf{Versione mobile}	
	\begin{figure}[!h]				  
		\centering
		\includegraphics[width=0.7\linewidth]{sezioni/Progettazione/Immagini/mobile_layout.png}
	    \caption{Layout di una pagina in versione mobile: menu chiuso(sinistra) e menu aperto(destra)}
	    \label{Fig:verMobile}
	\end{figure}	  
\end{itemize}
\newpage	
